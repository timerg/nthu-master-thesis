\thispagestyle{empty}
\chapter*{\centerline{Abstract}}
\addcontentsline{toc}{chapter}{Abstract}


Poly-silicon nanowire (SiNW) is a well-studied and interesting one-dimensional nanostructure.
Since it was introduced to the biosensor field in 2001, it has become a promising candidate for ultra-sensitive, real-time and label-free sensor device.
Nevertheless, many physical and chemical challenges constrain nanowire from being robust and practical.
Nowadays, many studies adopt the integrated-circuit techniques to solve the problems.
Circuits with different design concepts and purposes are proposed to meet practical needs.

In this thesis, based on the nanowire designed by Prof.Yang (National Chiao Tong University), we design our own read-out circuit.
This research first analyzes biological experiments results (From Prof.Yang) and the electrical characteristics of the nanowires.
The circuit specification and design is then based on these data analysis.

The circuit is capable of performing both DC-sweep ($I_D$-$V_G$ sweep) and transient measurement.
Moreover, we proposed a measurement method a combining of these two functions.
We believe this method mitigates the device variability induced by the fabrication process.
Currently, most operations in this method are manual.
We hope to make them automatic in the future by inducing digital circuits and constructing a system-level structure.





