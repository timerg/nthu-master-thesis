\chapter{Conclusion and Future Work}
\section{Conclusion}
\begin{figure}[!htb]
    \centering
    \includegraphics[width=0.7\textwidth] {images/chapter6/layout.PNG}
    \caption{The layout of the Readout circuit}
    \label{fig:layout}
\end{figure}
Fig.\ref{fig:layout} is the chip layout of the circuit.
It contains four unit of the read-out circuits and is able to measure four nanowire devices at the same time.

In this project, a circuit with two mode: DC-sweep mode and Transient Measurement mode is designed.
The first mode is to perform $I_D$-$V_G$ sweep while the second mode is to perform transient measurement.
By combining two mode, the circuit perform measurement by the variability-resisting method we proposed in this project.
This method mitigate the device variability problem and can be further improve in the future.
In Table.\ref{tb:ConcCompare}, our circuit is compared with the methods that were reviewed in chapter 2 \cite{Juv2}, \cite{Jlockin}.
Our circuit has wider $\Delta I$ detection range and lower power consumption when comparing with the similar work \cite{Jlockin}.
The work in \cite{Juv2} has better performance over all.
Still our circuit deals with the device variability problem, which does not be mentioned in both works.

\begin{table}[!htb]
    {\fontfamily{}\fontsize{10}{14}\selectfont
    \centering
    \begin{tabular}{l|c|c|c}
        & \cite{Juv2} & \cite{Jlockin} & This work \\
        \hline
        $\Delta I$ &  $120\mu A \sim 0.12n A$ & $3\mu A \sim 60n A$ &
                \begin{tabular}{@{}c@{}}
                    $3n A \sim 5.3\mu A$ \\ $-15\mu A\sim -3n A$
                \end{tabular}
                \\
        \hline
        Power consumption & 14.82uW & 2mW & 1.48mW\\
        \hline
        CMOS Technology & 0.13um & 0.18um & 0.35um\\
        \hline
        Device Variability & No Discussion & No Discussion & Variability-resisting method \\
    \end{tabular}
    \caption{Specification Summary}
    \label{tb:ConcCompare}
    }
\end{table}




\section{Future Work}
\paragraph{Problems in the Circuit Design}
The low current defect in DC-sweep mode is the most important problem that must be solved.
The solution for it is to replace the feedback OP with other closed-loop amplifier.
Another problem is the oscillation problem that happens when the $A_{amp}$ of the amplifier in the second-stage circuit is 1.
This problem can be solved simply by fixing the insufficient phase margin.

\paragraph{Introduce Filter and Better Experimental Process}
The noise included in Transient Measurement mode could be removed by simply introduce a bandpass filter.
The problem of this is that the signal speed is still hard to be defined.
Besides, currently we use pipetman to change the concentration of solution.
This process can evoke undesired noise and sometimes may not be carried on smoothly.
Both of these factors destroy or affect the speed of the signal.
If the process is improved and the signal speed is decided, a filter can be introduced and the noise can be reduced.

Furthermore, for the measurement dealing with the device variability problem, a method to decide when to switch between DC-sweep and Transient Measurement mode is needed.
This may be achieved by introducing digital circuit or by adding a feedback circuit for detecting whether the concentration changing reach a balance.

\paragraph{The $\Delta V$ Problem}
In Section.\ref{sec:ch6:dvp}, the method is functional but not perfect.
We have discussed the improvement methods.
They should help the further development of design and finally remove the device variability problem.
