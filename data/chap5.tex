\chapter{Integrated Circuitry Design}
This chapter presents the design of the read-out circuit and the simulation results.


\section{Architecture}
The review of source follower in section.\ref{section:SF} suggests the constant current method for the circuit of DC measurement.
The data analysis from chapter 3 supports it by the linear relation between $I_D$ and $g_m$.
However, the section.\ref{sec:AC} shows that source follower is not suitable for AC measurement.
It alternatively recommends the circuit in Fig.\ref{fig:lockin} which measures ac current signal and converts it into voltage output.
This circuit is appealing because of its noise suppression, simplicity and flexibility.

We combined these two method into one circuit structure and introduce them below.

\subsection{Circuit Schematic: First Stage}

\begin{figure}[!htbp]
    \centering
    \includegraphics[width=0.4\textwidth] {images/chapter5/DCMode.png}
    \caption{}
    \label{fig:DCmode}
\end{figure}

\paragraph*{DC mode}
Fig.\ref{fig:DCmode} is the first stage of our read-out circuit connected in DC mode.
As in the source follower, there is a bias current source (Ibias) for controlling the $I_D$.
The difference is that the Ibias inputs the current into drain instead of source.
In addition to this, we apply the TIA from section.\ref{sec:AC} (\cite{Jlockin}).
Its output connects to a rail-to-rail OP amplifier, which forms a negative feedback loop.
When the $I_D$ is greater than the current of Ibias, it lowers the output voltage of TIA and raise the gate voltage $V_G$.
And vice versa.
This is to say that the feedback mechanism forces the $I_D$ be equal to the current of Ibias by adjusting the $V_G$.

\begin{figure}[!htbp]
    \centering
    \begin{minipage}[t]{0.4\textwidth}
        \includegraphics[width=1\textwidth] {images/chapter5/ACMode.png}
        \raggedleft
        (a)
    \end{minipage}
    \hfill
    \begin{minipage}[t]{0.4\textwidth}
        \includegraphics[width=1\textwidth] {images/chapter5/ACMode_sine.png}
        \raggedleft
        (b)
    \end{minipage}
    \caption{}
    \label{fig:ACmode}
\end{figure}

\paragraph*{AC mode}
In Fig.\ref{fig:ACmode}(a), the switch in Fig.\ref{fig:DCmode} turns to a simple voltage source (Vg) which bias nanowire with the gate voltage.
The OP is nonfunctional with its output connected floating.
When performing measurement, we directly find how the solution concentration changes the output voltage ($V_{out}$).
This output voltage will input into the second stage circuit which is for the amplification.

\paragraph*{Dealing with the Disparity Problem}
As mentioned in chapter 1, we combine DC mode and AC mode to perform a disparity-resisting measurement.
Assuming there are two nanowire elements having disparity problem.
Initially, they are applied with DC mode to perform the $I_D$-$V_G$ sweeping.
We calculate their $g_m$ by the derivative of $I_D$ of $V_G$ and find the $g_m$-$I_D$ relation of these elements.
Then we select a $g_m$ value.
For the two elements, this value corresponds to two $I_D$.
And these two $I_D$ corresponds to two $V_G$.
We set the Ibias and Vg to these values.
Finally, after these initialization steps, we add the measuring solutions and detect the difference of $V_{out}$.
Since elements have same $g_m$, they should produce same voltage differences.
Before the next measuring solutions is added, we return to DC mode to find new $V_G$.
This reset $I_D$ of the elements to the value of Ibias, which also reset the $g_m$ to the value we selected.

\paragraph*{Another Usage of AC mode}
There is another way to perform measurement with AC mode, which resembles the method in \cite{Jlockin}.
This method measures the $g_m$ of nanowire.
As in Fig.\ref{fig:ACmode}(b), the source of nanowire is applied with an ac signal ($v_s$).
An ac output in $V_{out}$ is equal to $v_sg_m \times R_{TIA}$.
The values of Ibias and Vg can be arbitrary.
But one need to be aware that the values should not saturate the output of TIA or the second stage circuit.

Below, we talk about some design concept.

\subsubsection{Detecting Range Improvement (AC mode)} \label{sec:Ibias}
In section.\ref{sec:ch2CC} about the TIA subcricuit as Fig.\ref{fig:TIA_old}(a), we mentioned that the detecting range of $R_{NW}$ is limited.
There are same limits in our circuit which detects the current variance of nanowire.
We now discuss the causes of these limits and show the strategies we use.
To be noticed that the TIA block is a two-stage differential input and single output op-amp in the flowing discussion.

\begin{figure}[!htbp]
    \centering
    \begin{minipage}[t]{0.4\textwidth}
        \includegraphics[width=1.1\textwidth] {images/chapter5/TIA_olda.png}
        (a)
    \end{minipage}
    \hfill
    \begin{minipage}[t]{0.4\textwidth}
        \includegraphics[width=1\textwidth] {images/chapter5/TIA_old.png}
        (b)
    \end{minipage}
    \caption{\textbf{(a)} The transimpedance block (TIA) of the readout circuit from \cite{Jlockin}. The circuit input a voltage signal into resistive nanowire element $R_{NW}$.
            To compare it with our circuit (Fig.\ref{fig:TIA}), we transform the voltage input into an equivalent current input in \textbf{(b)}. The $I_{NW} = (V_{Ref} - V_{in})/R_{NW}$ and $\Delta i = \Delta vi /R_{NW}$}
    \label{fig:TIA_old}
\end{figure}

\paragraph*{Lower Limit:}
According to Fig.\ref{fig:TIA_old}(b), the TIA output voltage is:
\begin{equation}
    V_{TIA} = V_{Ref} + I_{NW}R_{TIA} + \Delta iR_{TIA}
    \label{eq:TIA_old}
\end{equation}
Two reasons which result in the lower limit of the detecting range relate to a large offset current $I_{NW}$.
One is that the output current provided by the TIA is restricted by design.
The other is that the restriction of the current flowing through the resistor $R_{TIA}$:
\begin{align}
    \frac{V_{Ref}}{R_{TIA}} < I_{NW} < \frac{VDD - V_{Ref}}{R_{TIA}}
\end{align}
Both reasons leads to the output saturation of TIA.

A naive way to handle the first reason is to increase the output current which TIA can provide.
The side effects of this method are the increases in power consumption and chip area.
As for the second reason, using smaller $R_{TIA}$ can ease the restriction on $I_{NW}$ and reduce the lower limit.
Unfortunately, it reduces the upper limit as well.
This will be discussed in the paragraph of ``upper limit''.

Our strategy for decreasing the lower limit is by utilizing the bias current source of nanowire.
According to Fig.\ref{fig:TIA}, the Eq.\ref{eq:TIA_old} is transformed into:
\begin{equation}
    V_{TIA} = V_{Ref} + (I_{NW} - I_{bias}) R_{TIA} + \Delta iR_{TIA}
    \label{eq:TIA}
\end{equation}
Now we can diminish the large $I_{NW}$ by Ibias.
In conclusion, the large offset current cause the saturation of the output of TIA.
We use the biasing current source to diminish that offset current, which increase the detecting range.

\begin{figure}[!htbp]
    \centering
        \includegraphics[width=0.4\textwidth] {images/chapter5/TIA.png}
    \caption{}
    \label{fig:TIA}
\end{figure}
\paragraph*{Upper Limit:}
The upper limit issues from the output resolution.
If the signal $\Delta iR_{TIA}$ from Eq.\ref{fig:TIA} is too small, it may be defeated by the noise.
One may be raise up the SNR by using large $R_{TIA}$.
However, the chip area constrain the size of resistors.
In our circuit design, we cannot make the resistance value out of $100k\Omega$.
Furthermore, even if the resistor can be greater, one need to concern for the noise brought by the large resistance.

Our strategy is to boost the SNR of TIA by designing its input MOSFETs in a large area.
And we also amplify the output signal through the second stage circuit.

\subsubsection{Input impedance Matching (DC mode)}
From the chapter 4, we learned that for the constant current method, the impedance matching between current source Ibias and nanowire element is important.
The matching decides the limit of the biasing current.
Here we calculate the input impedance of the circuit.

\begin{figure}[!htbp]
    \centering
        \includegraphics[width=0.4\textwidth] {images/chapter5/input_imp.png}
    \caption{}
    \label{fig:input_imp}
\end{figure}
In Fig.\ref{fig:input_imp}, by applying and input voltage $\Delta v_x$, we find the $\Delta i_x$.
The input impedance of the circuit is $\Delta v_x / \Delta i_x$.
\begin{align}
      \Delta i_x &= \quad i_D + i_{TIA} \\
      i_D &= \quad \frac{\Delta v_x}{r_{ds}} + \Delta v_xA_{TIA}A_{OP}g_m\\
      i_{TIA} &= \quad \frac {\Delta v_x}{R_{TIA} / (1 + A_{TIA})} \\
      \Delta v_x / \Delta i_x  &= \quad (\frac{1}{r_{ds}} + \frac{1 + A_{TIA})}{R_{TIA}})^{-1} \label{eq:input_imp}
\end{align}
The $A_{TIA}$ is the gain of the opAmp in TIA block.
The $A_{OP}$ is the gain of OP.
The $r_ds$ is the drain-to-source resistance of nanowire which is large than $100k\Omega$.
From the Eq.\ref{eq:input_imp}, we conclude that the input impedance of the circuit is equal to:
\begin{equation}
        \frac{1/g_m}{A_{TIA}\times A_{OP}}|| r_{ds} || \frac{R_{TIA}}{1 + A_{TIA})}
\end{equation}

In out design, the Ibias is an simple pmos.
Its output impedance roughly ranges from $1M\Omega$ to $1G\Omega$.
Since the $r_{ds}$ is large than $100k\Omega$.
The input impedance is clearly smaller than 1/100 fold of the $r_{ds}$, which means the impedance matching successes.

The result above seems implies that our design can achieve a lower biasing current than the circuit in chapter 4.
Unfortunately, it is not true.
A portion of the current given by Ibias leaks into the $R_{TIA}$.
We calculate the current ratio between $i_{TIA}$ and $i_D$:
\begin{align}
    \frac{i_D}{i_{TIA}} =  R_{TIA} \times g_m \times A_{OP}
\end{align}
Another section will shows that this value may not be over 10k (80dB) due to the stability.
Thus we can see that when $g_m$ is getting lower than $0.1\mu$, the ratio is smaller than 100 and the leakage current is not ignorant anymore.
This limit of the biasing current is same with the one of the circuit in chapter 4(Fig.\ref{fig:SF_imp}).

\subsection{Architecture: Second Stage}
We discuss the second stage circuit in this section.
To be noted that the second stage circuit is only for AC mode.

\begin{figure}[!htbp]
    \centering
        \includegraphics[width=1\textwidth] {images/chapter5/Second.png}
    \caption{}
    \label{fig:secondStage}
\end{figure}
Fig.\ref{fig:secondStage} is the block diagram of the second stage circuit.
The analog subtractor shifts the voltage of the $V_{in}$ from $V_{Ref}$ to $V_z$.
It follows by a resistor-based non-inverting amplifier composed of an two-stage differential OpAmp, two switches and three resistors.
The switches select the amplification rate among 100, 10 and 1.



\subsection{Design Spec and Calculation}













% with a transimpedance as front-end
% As stated in chapter 2,
