\chapter{Nanowire Structure and Measurement}
\section{Brief Description of Nanowire Structure}
The nanowire we use is made by Prof.Yang's team (National Chiao Tong University)\cite{J8}.
A sectional view of the nanowire structure is given below.
The fabrication process is based on the poly-silicon sidewall spacer technique.
The n-Type doped poly-SiNW FET has 2 to 10 poly-silicon channels.
Each channel is 80nm in width and 2µm in length.
Large portion of the channel surface is exposed to environment.
The exposed region, through several post-process, capture the DNA probe and serve as the sensing site for DNA molecules.\cite{C5, C6}

\begin{figure}[!htbp]
    \centering
    {\fontfamily{pag}\selectfont\textbf{
        \def\svgwidth{5.0cm}
        \fontsize{6}{7}\selectfont
        \input {images/drawing.pdf_tex}
    }}
    \fontsize{6}{7}\selectfont
    \caption{Nanowire Structure}
    \label{fig:res}
\end{figure}



\section{Measurement}

\subsection*{Front Gate and Back Gate}
Two gates are available: front-gate (liquid gate) and back-gate.
We choose front-gate as the operation gate in spite of some advantages that back-gate has.
One among those advantages is the ablility to lower the 1/f noise.\cite{C7}
However, this only happens in a very high gate voltage, which is not practical in the integrated circuit design.
Moreover, the front-gate induces larger drain-current.
In other words, it has higher transconductance. And a high transconductance leads to a stronger feedback ability in our design.




\subsection{External Factor and Experimental Protocol}
Several conditions effect nanowire performance.
According to Yang's team, the nanowire using thick gate dielectric and having non-regular cross-sectional
shape result in uncertainties of fabrication. \cite{C6}



 

