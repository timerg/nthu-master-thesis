\chapter{Nanowire Structure and Measurement}
\section{Brief Description of Nanowire Structure}
The nanowire we use is made by Prof.Yang's team (National Chiao Tong University)\cite{J8}.
A sectional view of the nanowire structure is given below.
The fabrication process is based on the poly-silicon sidewall spacer technique.
The n-Type doped poly-SiNW FET has 2 to 10 poly-silicon channels.
Each channel is 80nm in width and 2µm in length.
Large portion of the channel surface is exposed to environment.
The exposed region, through several post-process, capture the DNA probe and serve as the sensing site for DNA molecules.\cite{C5, C6}

\begin{figure}[!htbp]
    \centering
    {\fontfamily{pag}\selectfont\textbf{
        \def\svgwidth{5.0cm}
        \fontsize{6}{7}\selectfont
        \input {images/drawing.pdf_tex}
    }}
    \fontsize{6}{7}\selectfont
    \caption{Nanowire Structure}
    \label{fig:res}
\end{figure}



\section{Measurement}
This section presents the results.

\subsection*{Front Gate and Back Gate}
Two gates are available: front-gate (liquid gate) and back-gate.
We choose front-gate as the operation gate in spite of some advantages that back-gate has.
One of them is the ablility to lower the 1/f noise \cite{C7, C8}.
However, this only happens in a very high gate voltage, which is not practical in the integrated circuit design.
Moreover, the front-gate induces larger drain-current.
In other words, it has higher transconductance. And a high transconductance leads to a stronger feedback ability in our design.

% \begin{figure}[!htbp]
%     \centering
%     {\fontfamily{pag}\selectfont\textbf{
%         \def\svgwidth{5.0cm}
%         \fontsize{6}{7}\selectfont
%         \input {images/FgBg_Compare_Id_dev.pdf_tex}
%     }}
%     \fontsize{6}{7}\selectfont
%     \caption{}
%     \label{fig:res}
% \end{figure}
%
% \begin{figure}[!htbp]
%     \centering
%     {\fontfamily{pag}\selectfont\textbf{
%         \def\svgwidth{5.0cm}
%         \fontsize{6}{7}\selectfont
%         \input {images/FgBg_Compare_Id.pdf_tex}
%     }}
%     \fontsize{6}{7}\selectfont
%     \caption{}
%     \label{fig:res}
% \end{figure}


\begin{figure}[!htbp]
    \centering
    \begin{minipage}[t][0.1\textheight]{1\textwidth}
        \centering
        \def\svgwidth{10cm}
        \fontsize{6}{15}\selectfont
        \input {images/FgBg_Compare_Id.pdf_tex}
        (a)
    \end{minipage}
    \vfill
    \begin{minipage}[t][0.1\textheight]{1\textwidth}
        \centering
        \def\svgwidth{10cm}
        \fontsize{6}{15}\selectfont
        \input {images/FgBg_Compare_Id_dev.pdf_tex}
        (b)
    \end{minipage}
    \caption{}
    \label{fig:res}
\end{figure}

\subsection{Parameters}
The most crucial parameter for our circuit design is the transconductance (gm).
{\color{red}
    The gm is acquired by finding the relation between drain-to-source current ($I_d$) and gate-source voltage ($V_g$), and perform differentiation: $\frac{\partial I_d}{\partial V_g}$.
    use standard PBS as
}

\begin{figure}[!htbp]
    \centering
    \begin{minipage}[t][0.1\textheight]{1\textwidth}
        \centering
        \def\svgwidth{10cm}
        \fontsize{10}{20}\selectfont
        \input {images/pIdVg.pdf_tex}
        (a)
    \end{minipage}
    \hfill
    \begin{minipage}[t][0.1\textheight]{1\textwidth}
        \centering
        \def\svgwidth{10cm}
        \fontsize{10}{20}\selectfont
        \input {images/pIdgbs.pdf_tex}
        (b)
    \end{minipage}
    \caption{}
    \label{fig:res}
\end{figure}

The Id-Derivative figures indicates there is a ``linear region'' where gm is proportional to Id.
This property implies the transconductance can be controlled in simple way.
As mentioned in introduction, we may find specific bias Id for distinct elements and adjust their transconductance to a same value.

\begin{figure}[!htbp]
    \centering
    % {\fontfamily{pag}\selectfont\textbf{
        \def\svgwidth{10cm}
        \fontsize{10}{20}\selectfont
        \input {images/drawing-1.pdf_tex}
    % }}
    \fontsize{6}{7}\selectfont
    \caption{Distinct element with a line idicate they have same transconductance}
    \label{fig:res}
\end{figure}

We also prove that the transconductance under this region is unaffected by the drain-source voltage variance.

\begin{figure}[!htbp]
    \centering
    % {\fontfamily{pag}\selectfont\textbf{
        \def\svgwidth{10cm}
        \fontsize{10}{20}\selectfont
        \input {images/pIdgbs_Vd.pdf_tex}
    % }}
    \fontsize{6}{7}\selectfont
    \caption{Id-transcinductance with Vds variance}
    \label{fig:res}
\end{figure}











\subsection{External Factor and Experimental Protocol}
Several conditions effect nanowire performance.
According to Yang's team, the nanowire using thick gate dielectric and having non-regular cross-sectional
shape result in uncertainties of fabrication \cite{C6}.
Figure below shows that two elements lying on the same wafer can exhibit different electrical characteristics.

\begin{figure}[!htbp]
    \centering
    {\fontfamily{pag}\selectfont\textbf{
        \def\svgwidth{5.0cm}
        \fontsize{6}{7}\selectfont
        \input {images/drawing-1.pdf_tex}
    }}
    \fontsize{6}{7}\selectfont
    \caption{Nanowire Structure}
    \label{fig:res}
\end{figure}

 

