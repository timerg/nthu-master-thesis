\chapter{Literature Review \&}
As previously mentioned in the introduction section, the read-out circuit we proposed has two operation mode (LS and SS).
The LS mode is for drain current biasing while the SS mode is for current variance measurement.
Each of them references different sources.

\section{Constant Current and Source Follower}
In this section, we review the knowledge and an article that is related to our design of large signal mode (LS).

\subsection{Constant Current}


\subsection{Source Follower}
As one of the basic single stage amplifier, source follower (common drain) are employed to transfer voltage signal from gate to source while keeping drain current constant.
Although we haven't seen it is applied to nanowire, there have been several applications in the read-out circuits of ISFET (Ion-sensitive Field-effect Transistor)\cite{SF1, SF2} for a long time.

In ``Rapid Detection of E.coli Bacteria using Potassium-Sensitive FETs in CMOS", ISFET is applied as a biological transducer that convert detected bio-signal into it's input signal on the gate-end, which is resemble to our biosensor of nanowire.
An read-out circuit of source follower is served as the analog front-end.
The bio-signal induced voltage difference at the ISFET gate-end are converted to the source-end.
There is no need for an extra current-to-voltage converter which may import more signal fluctuation such as shot noise or flicker noise.
But on the other hand, the circuit requires a biasing current source.
This current source may have to be stable, noiseless or wide-range on demand.
And since the current value are usually under micro-scale even nano-scale, it is impractical to merely use external current source.
The article use two resistors and an op-amp to design a current scale down circuit.
Bias current decreases in proportional to the resistance ratio (N) of one resistor to another.
Moreover, by keeping Vds at a constant value (0.5v), the circuit also removes {\color{red}the channel affect which is a factor that may effect linearity of the results.}
It is showed in the schematic below that two op-amp based unit gain buffer are added to force the voltage at drain-end follows the source-end.

{\color{red} Schematic}

An issue need to be noticed is the impedance matching between the element and the current source circuit.
It is known that the output impedance of current source should be much larger than the input impedance of the biased element.
The equation for the output impedance of source follower is:
\setlength{\mathindent}{6.5cm}
\begin{align}
    & \frac{r_{ds}}{1 + g_mr_{ds}}\\
\intertext{The $g_m$ is the transconductance ($\frac{\partial I_d}{\partial V_{gs}}$) and the $r_{ds}$ is the drain-to-source impedance.
This equation can be simplified as:}
    & \frac{1}{g_m} \qquad \text{ for } \quad g_mr_{ds} >> 1 \label{eq:rsf2}
\end{align}
We also compute the output impedance of the current source circuit:
\begin{equation} \label{eq:rcs}
    N\times R_i
\end{equation}
$R_i$ is the impedance of the current source {\color{red}{I1 in Fig}}.
In the integrated circuit, $R_i$ is not ideal but usually close to the $r_{ds}$ of a single MOSFET.

As mentioned, Eq.\ref{eq:rcs} should be far larger than Eq.\ref{eq:rsf2}.
However, $g_m$ is proportional to the $I_d$, which means Eq.\ref{eq:rsf2} is inversely proportional to N.
When the bias current decreases, the output impedance decreases while the input impedance at the ISFET source-end increases.
This creates a lower boundary of the bias current.

The source follower structure provides a direct signal transition method.
It is a good candidate for the read-out circuit with the aim of detecting transconductance or threshold voltage variance.
Nevertheless, post-processing such as amplification and filtering are necessary.
The experiment results in the article are untreated.
Some strong signal attenuation exist, which are mainly caused by noise.

We constructed this circuit with discrete elements and applied it to our nanowire. The results are presented in chapter 4.
























\section{Data Analysis}
