\chapter{Literature Review \&}
As previously mentioned in the introduction section, the read-out circuit we proposed has two operation mode (DC and AC).
The DC mode is for drain current biasing while the AC mode is for current variance measurement.
Each of them references different sources.

\section{DC Sweep and Source Follower}
In this section, we review the knowledge and an article that is related to our design of large signal mode (DC).

\subsection{Constant Current}


\subsection{Source Follower}
{\color{red}{Fig of source follower}}
As one of the basic single stage amplifier, source follower (common drain) are employed to transfer voltage signal from gate to source while keeping drain current constant.
The transfer function can be derived as:
\setlength{\mathindent}{6.5cm}
\begin{align}
    \frac{V_{out}}{V_{in}} & = \frac{r_{ds}g_m}{1 + r_{ds}g_m} \\    \label{eq:sfTF}
                           & \approx 1 \qquad \text{ for } \quad r_{ds}g_m >> 1
\end{align}
$g_m$ is the transconductance ($\frac{\partial I_d}{\partial V_{gs}}$) and $r_{ds}$ is the drain-to-source impedance.
Although we haven't seen it is applied to nanowire, there have been several applications in the read-out circuits of ISFET (Ion-sensitive Field-effect Transistor)\cite{SF1, SF2} for a long time.

In ``Rapid Detection of E.coli Bacteria using Potassium-Sensitive FETs in CMOS", ISFET is applied as a biological transducer that convert detected bio-signal into it's input signal on the gate-end, which is resemble to our biosensor of nanowire.
An read-out circuit of source follower is served as the analog front-end.
The bio-signal induced voltage difference at the ISFET gate-end are converted to the source-end.
There is no need for an extra current-to-voltage converter which may import more signal fluctuation such as shot noise or flicker noise.
But on the other hand, the circuit requires a biasing current source.
This current source may have to be stable, noiseless or wide-range on demand.
And since the current value are usually under micro-scale even nano-scale, it is impractical to merely use external current source.
The article use two resistors and an op-amp to design a current scale down circuit.
Bias current decreases in proportional to the resistance ratio (N) of one resistor to another.
Moreover, by keeping Vds at a constant value (0.5v), the circuit also removes {\color{red}the channel affect which is a factor that may effect linearity of the results.}
It is showed in the schematic below that two op-amp based unit gain buffer are added to force the voltage at drain-end follows the source-end.

{\color{red} Schematic}

An issue need to be noticed is the impedance matching between the element and the current source circuit.
It is known that the output impedance of current source should be much larger than the input impedance of the biased element.
The equation for the output impedance of source follower is:

\begin{align}
    & \frac{r_{ds}}{1 + g_mr_{ds}}\\
\intertext{This equation can be simplified as:}
    & \frac{1}{g_m} \qquad \text{ for } \quad g_mr_{ds} >> 1 \label{eq:rsf2}
\end{align}
We also compute the output impedance of the current source circuit:
\begin{equation} \label{eq:rcs}
    N\times R_i
\end{equation}
$R_i$ is the impedance of the current source {\color{red}{I1 in Fig}}.
In the integrated circuit, $R_i$ is not ideal but usually close to the $r_{ds}$ of a single MOSFET.

As mentioned, Eq.\ref{eq:rcs} should be far larger than Eq.\ref{eq:rsf2}.
However, $g_m$ is proportional to the $I_d$, which means Eq.\ref{eq:rsf2} is inversely proportional to N.
When the bias current decreases, the output impedance decreases while the input impedance at the ISFET source-end increases.
This creates a lower boundary of the bias current.

The source follower structure provides a direct signal transition method.
It is a good candidate for the read-out circuit with the aim of detecting transconductance or threshold voltage variance.
Nevertheless, post-processing such as amplification and filtering are necessary.
The experiment results in the article are untreated.
Some strong signal attenuation exist, which are mainly caused by low-frequency noise and ISFET drift \cite{Drift}.
The drift problem are dealt with through some signal processing techniques while noise problems are left untreated.

We constructed this circuit with discrete elements and applied it to our nanowire. The results are presented in chapter 4.


\section{Small Signal (AC) Measurement Method Review}
In previous section, the source follower we mentioned exhibited compelling advantages as a signal processing structure of nano-device.
However, the structure overcomes obstacles when being applied to the small signal detection.
Parasitic capacitors and resistors can severely influence the results.
We take a simple MOSFET for example.

{\color{red}{Fig of parasitic mosfet}}

With the parasitic elements from the figure above are included, we modified the transfer function Eq.\ref{eq:sfTF}
\begin{align}
    \frac{r_{ds}(sC_{gs} + gm)}{1 + r_{ds}(gm + s(C_{gs}+C_s))}
\end{align}
The equation can be similar to Eq.\ref{eq:sfTF} which roughly equals to 1 as long as $C_s$ is far more smaller than $C_{gs}$.
Unfortunately, $C_s$ can be large since the output end of source follower usually connects to the next stage input or even the chip pad.
In that case, the parasitic capacitors may attenuate the signal as signal frequency increase.

Parasitic element is an issue and deserve more consideration.
{\color{red}Works} we present below combine capacitance measurement with nanowire measure.   %not sure how many works yet.


The measurement system for ZnO-nanowire based sensor array from \cite{Juv1} applies the Time-over-Threshold techniques to its read-out circuit (Fig.\ref{fig:tot1}).
The circuit alternatively charges an on-chip capacitor ($C_{int}$) with a constant current and discharges it through the nano-material resistance (nanowire).
An inverter with its output switches from on to off when the capacitor is charged to its input threshold voltage, and vice versa.
This behavior convert information of nanowire such as capacitance and resistance into time information.
Both $C_{int}$ and capacitance of nanowire ($C_{NW}$) (from drain to source) effect charging time and together with the drain-to-source resistance effect the discharging time.

The work presented in \cite{Juv1} doesn't have enough explanation about how do they interpret the capacitance and resistance information.
It simply mentioned that a microcontroller is responsible for these calculation.
Besides, the work lacks simulation and experiment of using complex elements as measure target.
Most of the results are measurement of using concrete resistor as the substitute for nanowire and regard the $C_{NW}$ as 0p.
The only nanowire experiment given at last doesn't has good performance.

The recent publicans \cite{Juv2} by the team is more elaborate.
In Fig.\ref{fig:tot2}, nanowire append between point A and B.
The charing current is able to be applied from Mp1 or Mp2, which is determined by the ``sel" signal with the aid from the MUXs.
This is simply mean to perform a reverse measurement and we ignore it by assume sel = 1 and point B is virtually ground.
Now, we can see that the circuit design concept is actually same.
The current charge both $C_{int}$ and $C_{NW}$.
When the voltage at A exceed the threshold voltage, the output switches to off and feedback to turn off the Mp1.
(To be noted that the inverter at the output satge in \cite{Juv1} is replaced by a schmitt trigger.)
Then the capacitor discharges through nanowire ($r_{ds}$).
The right-bottom plot in Fig.\ref{fig:tot2} defines $T_0$ as the charging time and $T_1$ as the discharging time.
The derivation of the $R_{NW}$ and $C_{NW}$ in the work can be simplified as:
\setlength{\mathindent}{2cm}
\begin{align}
                         C_{NW}            & = T_0 - C_{base}\\
                         R_{NW}            & = \frac{T_1R_{par}}{(C_{NW} + C_{base})R_{par} - T_1}\\
    \text{where} \qquad  R_{NW} || R_{par} & = \frac{T_1}{C_{NW} + C_{base}}
\end{align}
$C_{base}$ are the $C_{int}$ plus parasitic capacitance and $R_{par}$ the parasitic resistance.
These parasitic elements comes from the transistor in the integrated circuit block such as MUX and Mp.
It must be noted that we doesn't concern the hysteresis of the schmitt trigger here owing to simplicity.











\begin{figure}[!htbp]
    \centering
    {\fontfamily{pag}\selectfont\textbf{
        \def\svgwidth{5.0cm}
        \fontsize{6}{7}\selectfont
        \input {images/drawing-1.pdf_tex}
    }}
    \fontsize{6}{7}\selectfont
    \caption{Nanowire Structure}
    \label{fig:tot1}
\end{figure}
\begin{figure}[!htbp]
    \centering
    {\fontfamily{pag}\selectfont\textbf{
        \def\svgwidth{5.0cm}
        \fontsize{6}{7}\selectfont
        \input {images/drawing-1.pdf_tex}
    }}
    \fontsize{6}{7}\selectfont
    \caption{Nanowire Structure}
    \label{fig:tot2}
\end{figure}
 % detects and calculates both resistance and capacitance of ZnO-nanowire based sensor array through a read-out circuit and an external microcontroller.






\section{Data Analysis}
