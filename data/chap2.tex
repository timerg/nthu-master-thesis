\chapter{Literature Review \&}
As previously mentioned in the introduction section, the read-out circuit we proposed has two operation mode (DC and AC).
The DC mode is for drain current biasing while the AC mode is for current variance measurement.
Each of them references different sources.

\section{DC Sweep and Source Follower}
In this section, we review the knowledge and an article that is related to our design of large signal mode (DC).

\subsection{Id-Vg Sweep and Transconductance}
A common method for examining nanowire electrical properties is to perform DC sweep.
In \cite{DN17}, two typical dc sweep experiments of nanowire at room temperature are given: $I_d$-$V_d$ and $I_d$-$V_g$.
From the direct observation of the experiment results, nanowire seems to be more ``sensitive" to $V-g$ than $V_d$ because changing $V_g$ induces more $I_d$ variance than changing $V_d$.
The cause is that in most of time, the transconductance ($g_m = \frac{\partial I_d}{\partial V_g}$) of a transistor is larger than its drain conductance ($g_d =  \frac{\partial I_d}{\partial V_d}$).
{\color{red}We thus surmise that using transconductance as may resolution }
% 需要確認清楚,濃度改變究竟是讓 transconductance 改變,還是成為另一個電壓差(或是threshold voltage 改變?)

% With the MOSFET model of weak and strong inversion, we have the $I_d$ equations of MOSFET:
% \begin{align}
%     \text{strong inversion:} \quad & I_d  = \mu C_{ox} \frac{W}{L}((V_{gs} - V_t)V_{ds} - \frac{V_{ds}^2}{2})\\
%     \text{weak inversion:} \quad & I_d  = I_0e^{\kappa V_{gs}/\phi_t}(1 - e^{-V_{ds}/\phi_t})
% \end{align}
% We can also derive the $g_m$ and $g_d$:
% \begin{align}
%     \text{strong inversion:} \quad & g_m  = \sqrt{2 \mu C_{ox} (\frac{W}{L})I_{dsat}\\
%                                  & g_d = \\
%     \text{weak inversion:} \quad & I_d  = I_0e^{\kappa V_{gs}/\phi_t}(1 - e^{-V_{ds}/\phi_t})
% \end{align}








\subsection{Source Follower}

\begin{figure}[h]
    \centering
    \includegraphics[width=5cm]{images/chapter2/SourceFollower.png}
    \fontsize{6}{7}\selectfont
    \caption{Sorce Follower}
    \label{fig:SF}
\end{figure}

As one of the basic single stage amplifier, source follower (common drain) are employed to transfer voltage signal from gate to source while keeping drain current constant.
The transfer function can be derived as:
\setlength{\mathindent}{5.5cm}
\begin{align}
    \frac{V_{out}}{V_{in}} & = \frac{r_{ds}g_m}{1 + r_{ds}g_m} \\    \label{eq:sfTF}
                           & \approx 1 \qquad \text{ for } \quad r_{ds}g_m >> 1
\end{align}
$g_m$ is the transconductance ($\frac{\partial I_d}{\partial V_{gs}}$) and $r_{ds}$ is the drain-to-source resistance.
Although we haven't seen it is applied to nanowire, there have been several applications in the read-out circuits of ISFET (Ion-sensitive Field-effect Transistor)\cite{SF1, SF2} for a long time.



In \cite{SF1}, ISFET is applied as a biological transducer that convert detected bio-signal into it's input signal on the gate-end, which is resemble to our biosensor of nanowire.
An read-out circuit of source follower is served as the analog front-end.
The bio-signal induced voltage difference at the ISFET gate-end are converted to the source-end.
There is no need for an extra current-to-voltage converter which may import more signal fluctuation such as shot noise or flicker noise.
But on the other hand, the circuit requires a biasing current source.
This current source may have to be stable, noiseless or wide-range on demand.
And since the current value are usually under micro-scale even nano-scale, it is impractical to merely use external current source.
The article use two resistors and an op-amp to design a current scale down circuit.
Bias current decreases in proportional to the resistance ratio (N) of one resistor to another.
Moreover, by keeping Vds at a constant value (0.5v), the circuit also removes {\color{red}the channel affect which is a factor that may effect linearity of the results.}
It is showed in the schematic below that two op-amp based unit gain buffer are added to force the voltage at drain-end follows the source-end.

\begin{figure}[h]
    \centering
    \includegraphics[width=12cm]{images/chapter2/ISFET.png}
    \fontsize{6}{7}\selectfont
    \caption{ISFET readout circuit in \cite{SF1}}
    \label{fig:ISFET}
\end{figure}

An issue need to be noticed is the impedance matching between the element and the current source circuit.
It is known that the output impedance of current source should be much larger than the input impedance of the biased element.
The equation for the output impedance of source follower is:

\begin{align}
    & \frac{r_{ds}}{1 + g_mr_{ds}}\\
\intertext{This equation can be simplified as:}
    & \frac{1}{g_m} \qquad \text{ for } \quad g_mr_{ds} >> 1 \label{eq:rsf2}
\end{align}
We also compute the output impedance of the current source circuit:
\begin{equation} \label{eq:rcs}
    N\times R_i
\end{equation}
$R_i$ is the impedance of the right-bottom current source in Fig.\ref{fig:ISFET}.
In the integrated circuit, $R_i$ is not ideal but usually close to the $r_{ds}$ of a single MOSFET.

As mentioned, Eq.\ref{eq:rcs} should be far larger than Eq.\ref{eq:rsf2}.
However, $g_m$ is proportional to the $I_d$, which means Eq.\ref{eq:rsf2} is inversely proportional to N.
When the bias current decreases, the output impedance decreases while the input impedance at the ISFET source-end increases.
This creates a lower boundary of the bias current.

The source follower structure provides a direct signal transition method.
It is a good candidate for the read-out circuit with the aim of detecting transconductance or threshold voltage variance.
Nevertheless, post-processing such as amplification and filtering are necessary.
The experiment results in the article are untreated.
Some strong signal attenuation exist, which are mainly caused by low-frequency noise and ISFET drift \cite{Drift}.
The drift problem are dealt with through some signal processing techniques while noise problems are left untreated.

We constructed this circuit with discrete elements and applied it to our nanowire. The results are presented in chapter 4.


\section{Small Signal (AC) Measurement Method Review}
In previous section, the source follower we mentioned exhibited compelling advantages as a signal processing structure of nano-device.
However, the structure overcomes obstacles when being applied to the small signal detection.
Parasitic capacitors and resistors can severely influence the results.

\begin{figure}[h]
    \centering
    \includegraphics[width=5cm]{images/chapter2/SourceFollower_wi_parasiticCap.png}
    \fontsize{6}{7}\selectfont
    \caption{Sorce Follower with parasitic capacitance}
    \label{fig:SF_pC}
\end{figure}

As in figure \ref{fig:SF_pC} where the parasitic elements are included, we modified the transfer function Eq.\ref{eq:sfTF}
\begin{align}
    \frac{V_{out}}{V_{in}} = \frac{r_{ds}(sC_{gs} + gm)}{1 + r_{ds}(gm + s(C_{gs}+C_s))}
\end{align}
The equation can be similar to Eq.\ref{eq:sfTF} which roughly equals to 1 as long as $C_s$ is far more smaller than $C_{gs}$.
Unfortunately, $C_s$ can be large since the output end of source follower usually connects to the next stage input or a pad.
In that case, the parasitic capacitors may attenuate the signal.

We want to build another circuit structure that can not only performs ac signal measurement but also immunes from parasitic capacitance affect.
This is achieved by reviewing those works that try to measure the parasitic capacitance first.
Below, the works from two teams aims to measure drain-to-source resistance ($R_{NW}$) and drain-to-source capacitance ($C_{NW}$).
Base on the reviews, we adopted and modified the method from one of it (2.2.2).
This will be described briefly in section 2.2.3 and thoroughly in chapter 5.

\subsection{RC Time Delay Measuring}
The measurement system for ZnO-nanowire based sensor array from \cite{Juv1} applies the Time-over-Threshold techniques to its read-out circuit (Fig.\ref{fig:tot1}).
The circuit alternatively charges an on-chip capacitor ($C_{int}$) with a constant current and discharges it through the nano-material resistance (nanowire).
An inverter with its output switches from on to off when the capacitor is charged to its input threshold voltage, and vice versa.
This behavior convert information of nanowire such as capacitance and resistance into time information.
Both $C_{int}$ and $C_{NW}$ effect charging time and together with the $R_{NW}$ effect the discharging time.

The work presented in \cite{Juv1} doesn't have enough explanation about how do they interpret the capacitance and resistance information.
It simply mentioned that a microcontroller is responsible for these calculation.
Besides, the work lacks simulation and experiment of using complex elements as measure target.
Most of the results are measurement of using concrete resistor as the substitute for nanowire and regard the $C_{NW}$ as 0p.
The only nanowire experiment given at last doesn't has good performance.
It seems that the design may only be applied to a complete-resistor or complete-capacitor element.

The recent publicans \cite{Juv2} by the team is more elaborate and have measurement of complex element (An element composed of both resistor and capacitor).
In Fig.\ref{fig:tot2}, nanowire append between point A and B.
The charing current is able to be applied from Mp1 or Mp2, which is determined by the ``sel" signal with the aid from the MUXs.
This is simply mean to perform a reverse measurement and we ignore it by assume sel = 1 and point B is virtually ground.
Now, we can see that the circuit design concept is actually same.
The current charge both $C_{int}$ and $C_{NW}$.
When the voltage at A exceed the threshold voltage, the output switches to off and feedback to turn off the Mp1.
(To be noted that the inverter at the output satge in \cite{Juv1} is replaced by a schmitt trigger.)
Then the capacitor discharges through nanowire ($r_{ds}$).
The right-bottom plot in Fig.\ref{fig:tot2} defines $T_0$ as the charging time and $T_1$ as the discharging time.
The derivation of the $R_{NW}$ and $C_{NW}$ in the work can be simplified as:
\setlength{\mathindent}{2cm}
\begin{align}
                         C_{NW}            & = T_0 - C_{base}\\
                         R_{NW}            & = \frac{T_1R_{par}}{(C_{NW} + C_{base})R_{par} - T_1}\\
    \text{where} \qquad  R_{NW} || R_{par} & = \frac{T_1}{C_{NW} + C_{base}}
\end{align}
$C_{base}$ are the $C_{int}$ plus parasitic capacitance and $R_{par}$ the parasitic resistance.
These parasitic elements comes from the transistor in the integrated circuit block such as MUX and Mp.
It must be noted that owing to simplicity, we doesn't concern the hysteresis of the schmitt trigger here.

\begin{figure}[!htbp]
    \begin{minipage}[t][0.1\textheight]{0.6\textwidth}
        \centering
        \includegraphics[width=8cm]{images/chapter2/uv1.png}
        \caption{(a) Schematic of \cite{Juv1}}
        \label{fig:tot1}
    \end{minipage}
    \hfill
    \begin{minipage}[t][0.1\textheight]{0.4\textwidth}
        \centering
        \includegraphics[width=8cm]{images/chapter2/uv2.png}
        \caption{(b) Schematic of \cite{Juv2}}
        \label{fig:tot2}
    \end{minipage}
    \fontsize{6}{7}\selectfont
\end{figure}

\begin{figure}[!htbp]
    \centering
    {\fontfamily{pag}\selectfont\textbf{
        \def\svgwidth{5.0cm}
        \fontsize{6}{7}\selectfont
        \input {images/drawing-1.pdf_tex}
    }}
    \fontsize{6}{7}\selectfont
    \caption{Draw mos with (Cgd + Cd) and rds is modeled by Rnw and Cnw}
    \label{fig:tot_mos}
\end{figure}

\subsection{Complex Impedance Solving}
The nanowire-based hydrogen sensor measurement system from \cite{Jlockin} adopt another method.
It use a lock-in amplifier to realize both resistive and capacitive impedance measurement.

\begin{figure}[!htbp]
        \centering
        \includegraphics[width=13cm]{images/chapter2/lockin.png}
        \caption{(b) Block diagram of the lock-in amplifier in \cite{Jlockin}}
        \label{fig:lockin}
\end{figure}

The system started by supplying nanowire with a sinusoidal voltage signal to one end while the other end is grounded virtually by a transimpedance amplifier (TIA).
The TIA then converts the output current of nanowire into voltage signal which contains complex impedance information.
The resistance is in the real part while the capacitance is in the virtual part
\setlength{\mathindent}{5.5cm}
\begin{align}
    V_{out} &= I_{NW}R_{TIA} \\
    I_{NW} &= V_{in}(\frac{1}{R_{NW}} + j 2\pi fC_{NW})
\end{align}
$f$ is the frequency of input signal.

After remove high-order harmonic interferences by a controllable bandpass filter, the signal is demodulated.
The resistive and capacitive impedance values are resolved through channel I and Q with phase different by 90 degree.
The mixer is a linear multiplier that use for demodulation.
With a radio frequency (RF) input and the input local oscillator (LO) input, it produce an output signal that consists of signals with frequencies $f_{RF} + f_{LO}$ and $f_{RF} - f_{LO}$.
Incidentally, the signal are immune from the perturbation of low frequency noise which is a common problem for biosensor.

\subsection{Comparison and Conclusion}
We compare method 1 (Sec.2.2.1) and method 2 (Sec.2.2.2) here.
Both of them focus on detecting the $R_{NW}$ difference.
According to the comparison table below (\ref{tb:LVtable}), we can see the resistor measurement range of method 1 is different from 2 by a large extent.
This may because the minimum bias current of nanowire provided by the circuits are different.
The minimum current in method 1 is limited by the pmos(I charge) and the leakage current.
In method 2, it is limited by the TIA.
Since our method adopt this TIA block, we will discuss this problem in chapter 5.

As for the $C_{NW}$ detection,
the measure range is much worse.{\color{red}{Reason}}

Method 2 perform well when it comes to noise suppression.
In fact, the circuit in method 1 doesn't provide noise reduction ability.
The special structure it use (The article \cite{Juv1} mentioned it as M4N approach) is the one responsible for that.

Method 1 has a lower power consumption. However, it is under estimated since the microcontroller power is not included.

\begin{table}[!htb]
    {\fontfamily{}\fontsize{10}{14}\selectfont
    \centering
    \label{tb:LVtable}
    \begin{tabular}{l|cp{4cm}}
        & \cite{Juv2} & \cite{Jlockin}\\
        \hline
        R meas range & 1M - 1G & 10 - 40k\\
        \hline
        R meas error & < 2.5\% & < 2\%\\
        \hline
        C meas range & 100fF - 1uF & 0.5 - 1.8nF\\
        \hline
        C meas error & < 3\% & < 3\%\\
        \hline
        SNR & > 45dB & - \\
        \hline
        Input refered noise & - & 190 nV/sqrt(Hz) @ 5 kHz \\
        \hline
        CMOS Technology & 0.13um & 0.18um\\
        \hline
        Power consumption & 14.82uW & 2mW\\
    \end{tabular}
    \caption{Specification Summary}
    }
\end{table}

In our project, capacitance measurement is not our object.
But we will still need to consider the parasitic capacitor effect in our circuit design.
Method 1 convert the resistance information into time (frequency) information.
If one want to avoid the affect from parasitic capacitor, he should apply a $C_{int}$ that is much larger than $C_{NW}$.
However, it is not practical in integrated design because the chip size is limited.

Method 2 uses a TIA to measure resistance and capacitance together first and then resolve the complex value.
We notice that the capacitance value is much larger than the resistance value.
This difference may be revised downward in our silicon nanowire case.
The resistance can be more than 100M.
However, since $C_{NW}$ is parallel to $R_{NW}$, we wonder the C value can be ignored.
Besides, our circuit measure the $g_m$ instead of $R_{NW}$.
The value can be smaller than $R_{NW}$.
In fact, we can even control the $I_d$ of nanowire to lower the $g_m$ further.

Another reason that make method 2 more attractive is because the method is more flexible.
One can simply add other analog blocks such as noise filter or amplifier to it.

Overall, method 1 has advantage in detecting range and accuracy while method 2 has better noise suppression and flexibility.










%
