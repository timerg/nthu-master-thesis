\chapter{Introduction}
\section{Motivation}
Poly-silicon nanowire(SiNW) is an interesting and promising one-dimensional nano-structures.
Many research of fabrication and electrical properties have been conducted \cite{J1}.
It was first introduced to the biosensor field in 2001\cite{J2} and has become a promising candidate for various features such as high surface-to-volume ratio, ultra sensitivity, label-free electrical detection and  real-time measurement.

Although there has been some great advances on nanowire structure design \cite{J3}, the work of systems-level engineering is still insufficient.
Systems designed for specific purpose can help the device to meet practical needs.
{\color{red}Such as low noise }
Moreover, there are still several challenges that may be overcome through a better signal acquisition system \cite{R1}.

One of the challenges is that the mass production of robust nanowire is still improbable.
Element disparity may be a main reason among others.
This problem also happens to the measurement of our own nanowire (Fig.~\ref{fig:disparity}).
The nanowire we use is made by Professor Yang's team (National Chiao Tong University).
And according to them, the nanowire use thick gate dielectric and have non-regular cross-sectional shape, which result in uncertainties of fabrication \cite{C6}.


\section{Introduction}
In this project, we design a nanowire readout circuit that is capable of performing both large signal and small signal measurement.
Specifically, the circuit has two operation mode.
One is large signal mode (LS), which bias nanowire under a drain current ($I_d$) and finds corresponding gate-to-source voltage ($V_{gs}$).
The other is the small signal mode (SS).
It detects and amplifies the current variance of nanowire.
We also combines these two function to try a method for solving the disparity problem.

\subsection*{Method for solving the disparity problem}
It base on two assumptions.
\begin{enumerate}
    \item The nanowire transconductance is proportional to the drain-to-source current (Id).
    \item The changing of the concentration of targeted biomolecule can be viewed as the small signal voltage input to the gate end of a transistor.
\end{enumerate}
The first assumption implies one can control the nanowire transconductance by the biasing Id.
The second assumption means that as long as different nanowire elements have a same transconductance, they produce same output current (small signal).
We testify these assumption in chapter 3.

We implement the method through our read-out circuit with large signal mode (LS) and small signal mode (SS).
In the beginning of each measurement event, we applied the LS mode to initialized the drain current and gate voltage of nanowire.
This is to standardize the transconductance of every nanowire.
With nanowire biased under these $I_d$ and $V_{gs}$, the circuit turn to the SS.

Some minutiae is reviewed in chapter 5.
Currently the operation in the first mode is fully manual.
In the future work this needs to be modified and may requires digital circuit assistance.


\section{Design Flow and Chapter Layout}
There are six chapters in this thesis, which are sorted according to the design flow.

Chapter 2 are divided into two part.
The first part is the literature review. {\color{red}...} The other is the analysis of measurement data from Yang's team.
Most of those are the drain current of nanowire sweeping along the gate voltage (Id-Vg curves).
We present some of the raw data and the analysis results in this part.

Chapter 3 gives a brief description of nanowire structure.
It is then followed by our own nanowire measurement based on the information in the previous chapter.
The measurement includes:

\begin{enumerate}
\item Comparison between front gate and back gate
\item Nanowire transconductance
\item Nanowire drain-to-source resistance
\end{enumerate}

Chapter 4 is an ``accessory".
We construct an discrete circuit which was designed for ion-sensitive field-effect transistor (ISFET) \cite{J4}.
The purpose of this process is to practice the constant current method.
The outcomes are deficient and it is its reference value which we spotlight.

Chapter 5


\begin{figure}[!htbp]
    \centering
    {\fontfamily{pag}\selectfont\textbf{
        \def\svgwidth{5.0cm}
        \fontsize{6}{7}\selectfont
        \input {images/drawing-1.pdf_tex}
    }}
    \fontsize{6}{7}\selectfont
    \caption{Design Flow}
    \label{fig:designFlow}
\end{figure}



\section{Contribution to Knowledge}
Our
% This circuit has two operation mode.
% One is the ``Gate Voltage Tracing (GVT)" mode, and the other is the ``Current Variance Measure (CVM)" mode.



%
% {\color{red}A better measure method}
%
%  The concentration changing of the testing solution effect the intrinsic characteristics of nanowire.
% There are a variety of method for quantifying the affect.
% The most common ones are finding drain-source current and nanowire drain-to-source resistance.
% The method that Yang's team employed is to find several Id-Vg sweeping curves and see the rising/falling trend of them (Fig.~\ref{fig:IdVgandgbsId} (a)).
%  By the method of Yang's team, we had an intuition using transconductance instead of drain-to-source resistance may give a more {\color{red}usful} result.
% {\color{red}Because the }
%
%
% In the beginning, we adopted Yang's method and made some modification to it.
% By finding the derivative of Id with respect to Vg, we have the relation between Id and transconductance of nanowire (Fig.~\ref{fig:IdVgandgbsId} (b)).

% In our biosensor system, nanowire is treated as a MOSFET.
% Its gate bias under a specific voltage source
% And the bio-signal is viewed as small voltage signal input to the gate.
%
%
%  with its drain-source current ($I_{ds}$)  biased by a pmos current source.
% When a measurement event happens (such as a DNA concentration variation), the transconductance of nanowire changes and induces a current variance.
% This variance is converted into an amplified voltage signal.
% After the measurement, a feedback circuit pulls up/down the nanowire gate-source voltage ($V_{gs}$) to set $I_{ds}$ to the initail value.




%
% The different concentration of the measuring solution induce the change of nanowire transcconduction.
% By keeping bias voltage (drain, source, gates) same, the transconduction chage directly effect the drain-source current.
% The current variance signal is converted into voltage signal and is amplified.
% After each measurement, the drain-source current of nanowire returns to the constant value by a feedback circuit, which adjusts the gate-source voltage.
% the current variance induced by the difference of the solution concentration
% The other is the process variation. Since



