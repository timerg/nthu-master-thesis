\chapter{Introduction}
\section{Motivation}
Poly-silicon nanowire(SiNW) is a well-studied and promising one-dimensional nanostructure.
It was first introduced to the biosensor field in 2001\cite{C2001} and has become a potential candidate for various features such as high surface-to-volume ratio, ultra sensitivity, label-free electrical detection and real-time measurement.

Although there have been substantial advances on nanowire structure design \cite{R1}, the work on the system-level engineering is still insufficient.
Systems designed for a specific purpose can help the device to meet practical needs such as noise reduction, real-time measurement, and analog-to-digital conversion.
Moreover, there are still several challenges that may be overcome through a better signal acquisition system \cite{R1}.

One of the challenges is that the mass production of robust nanowire device is still improbable.
Device variability may be the main reason among others.
This problem also happens to the measurement of our nanowire.
The nanowire we use is made by Professor Yang's team (National Chiao Tong University).
According to them, the nanowire uses thick gate dielectric and has non-regular cross-sectional shape, which result in the problem of fabrication uncertainty \cite{C6}.


\section{Design Overview}
In this project, we design a nanowire read-out circuit with two modes: DC-sweep mode and Transient Measurement mode.
In DC-sweep mode, one can use the circuit to perform a DC sweep of drain-to-source current ($I_D$) to show how the gate voltage ($V_G$) changes, or gives nanowire a constant $I_D$ and measures the $V_{G}$ response to different solution concentration.
In Transient Measurement mode, the circuit detects and amplifies the variance of $I_D$ with constant bias voltages applied ($V_D$, $V_G$, $V_S$).
We also combine two modes to implement our proposal: the variability-resisting method.
This method of measurement may mitigate the device variability problem.

\subsection*{Dealing with the Device Variability Problem: the Variability-resisting Method} \label{section:twqAssumption}
The variability-resisting method is based on two assumptions:
\begin{enumerate}
    \item The nanowire transconductance ($g_m = \frac{\partial I_D}{\partial V_{GS}}$) is dependent on $I_D$ and independent of $V_{GS}$.
    \item The change of biomolecule concentrations can induce potential change on the surface of the gate of a nanowire.
\end{enumerate}
(Since the source of nanowire is always grounded in this method, from now on $V_{GS}$ is simplified as $V_G$.)
The first assumption implies one can control the nanowire transconductance by its $I_D$.
The second assumption means that as long as different nanowire elements have the same transconductance, the $I_D$ variance induced by a concentration difference should be same.

The method works as follows:

\paragraph*{Initial stage}
At the beginning of each measurement event, we perform a DC sweep with DC-sweep mode.
By analyzing the sweep results with numerical method, we keep all nanowire devices under a selected transconductance by controlling their $I_D$ and corresponding $V_G$.


\paragraph*{Measurement stage}
We bias the circuit in Transient Measurement mode at this stage.
Since the transconductance of all devices are same, they should behave uniformly based on assumption 2.
At the end of the stage, we return to DC-sweep mode to reset $I_D$ of the elements.
The circuit adjusts their $V_G$ to do so.

At the beginning of each measurement stage, a device always has the same $I_D$ but different $V_G$.
Based on assumption 1, its transconductance is kept constant.

Other details are reviewed in chapter 5.
Currently, most operations are manual.
We hope to make them automatic in the future, which may require digital circuits.

\section{Design Flow and Chapter Layout}
In this thesis, there are six chapters sorted according to the design flow.

Chapter 2 reviews the basic theories and the literature that are related to our work.

Chapter 3 gives a brief description of nanowire structure.
It is then followed by two sections about some measurement and data analysis.
The data of the first one is from the biological experiments while the second one is from the electrical measurement.
We use the analysis results to design the read-out circuit.

Chapter 4 is an ``accessory''.
This chapter contains the discrete circuit which was designed for ion-sensitive field-effect transistor (ISFET) \cite{SF1}.
It is constructed and some electrical measurements are performed
The purpose of this process is to practice the constant-current constant-voltage method.
The outcomes of this chapter underpin our integrated circuit design.

Chapter 5 talks about the schematic, design process and the simulation results of the read-out circuit.

Chapter 6 presents the measurement results of our integrated circuit and the conclusion of this project.





% This circuit has two operation mode.
% One is the ``Gate Voltage Tracing (GVT)" mode, and the other is the ``Current Variance Measure (CVM)" mode.



%
% {\color{red}A better measure method}
%
%  The concentration changing of the testing solution effect the intrinsic characteristics of nanowire.
% There are a variety of method for quantifying the affect.
% The most common ones are finding drain-source current and nanowire drain-to-source resistance.
% The method that Yang's team employed is to find several Id-Vg sweeping curves and see the rising/falling trend of them (Fig.~\ref{fig:IdVgandgbsId} (a)).
%  By the method of Yang's team, we had an intuition using transconductance instead of drain-to-source resistance may give a more {\color{red}usful} result.
% {\color{red}Because the }
%
%
% In the beginning, we adopted Yang's method and made some modification to it.
% By finding the derivative of Id with respect to Vg, we have the relation between Id and transconductance of nanowire (Fig.~\ref{fig:IdVgandgbsId} (b)).

% In our biosensor system, nanowire is treated as a MOSFET.
% Its gate bias under a specific voltage source
% And the bio-signal is viewed as small voltage signal input to the gate.
%
%
%  with its drain-source current ($I_{ds}$)  biased by a pmos current source.
% When a measurement event happens (such as a DNA concentration variation), the transconductance of nanowire changes and induces a current variance.
% This variance is converted into an amplified voltage signal.
% After the measurement, a feedback circuit pulls up/down the nanowire gate-source voltage ($V_{gs}$) to set $I_{ds}$ to the initail value.




%
% The different concentration of the measuring solution induce the change of nanowire transcconduction.
% By keeping bias voltage (drain, source, gates) same, the transconduction chage directly effect the drain-source current.
% The current variance signal is converted into voltage signal and is amplified.
% After each measurement, the drain-source current of nanowire returns to the constant value by a feedback circuit, which adjusts the gate-source voltage.
% the current variance induced by the difference of the solution concentration
% The other is the process variation. Since



