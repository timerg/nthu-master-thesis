\chapter{Introduction}
\section{Motivation}
Poly-silicon nanowire(SiNW) is an interesting and promising one-dimensional nano-structures.
Many research of fabrication and electrical properties have been conducted \cite{C25th}.
It was first introduced to the biosensor field in 2001\cite{C2001} and has become a promising candidate for various features such as high surface-to-volume ratio, ultra sensitivity, label-free electrical detection and  real-time measurement.

Although there has been some great advances on nanowire structure design \cite{R1}, the work of systems-level engineering is still insufficient.
Systems designed for specific purpose can help the device to meet practical needs such as noise reduction, real-time measuring and conversion to digital output.
Moreover, there are still several challenges that may be overcome through a better signal acquisition system \cite{R1}.

One of the challenges is that the mass production of robust nanowire is still improbable.
Element disparity may be a main reason among others.
This problem also happens to the measurement of our own nanowire (Fig.~\ref{fig:disparity}).
The nanowire we use is made by Professor Yang's team (National Chiao Tong University).
And according to them, the nanowire use thick gate dielectric and have non-regular cross-sectional shape, which result in uncertainties of fabrication \cite{C6}.


\section{Introduction}
In this project, we design a nanowire readout circuit with two modes that perform large signal (DC) and small signal (AC) measurement.
In DC mode, one can use the circuit to perform a DC sweep of drain current ($I_D$) to show how the gate voltage ($V_{G}$) changes, or gives nanowire a constant $I_D$ and measures the $V_{G}$ response to different solution concentration.
In AC mode, the circuit detects and amplifies the current variance of nanowire with constant bias voltages applied ($V_D$, $V_G$, $V_S$).
We also combine two modes to implement a proposed method that may mitigate the disparity problem.

\subsection*{Dealing with the disparity problem}
The proposed method base on two assumptions.
\begin{enumerate}
    \item The nanowire transconductance ($g_m = \frac{\partial I_D}{\partial V_{GS}}$) depends on $I_D$ and independent on $V_{GS}$.
    \item The changing of the concentration of bio-molecule can be viewed as a voltage signal input to the gate end of a transistor.
\end{enumerate}
The first assumption implies one can control the nanowire transconductance by the biasing Id.
The second assumption means that as long as different nanowire elements have a same transconductance, the output current induced by a concentration difference should be same.

The method works as follows:

\textbf{Initial stage}: In the beginning of each measurement event, we set circuit under DC mode and perform a DC sweep.
By handling the sweep results with numerical method, we keep all nanowire elements under a selected transconductance by controlling their $I_D$ and corresponded $V_G$.

\textbf{Measurement stage:} This is where the AC mode is used.
Since the transconductance of all elements are same, they should behave uniformly based on assumption 2.
In the end of the stage, we return to DC mode to reset $I_D$ of the elements.
The circuit adjusts their $V_G$ to do so.

In the beginning of each measurement stage, an element always has a same $I_D$ but different $V_G$.
Based on assumption 1, its transconductance is kept constant.


Some minutiae is reviewed in chapter 5.
Currently, most operations are manual.
We hope to make them automatic in the future, which may require digital circuit assistance.


\section{Design Flow and Chapter Layout}
There are six chapters in this thesis, which are sorted according to the design flow.

Chapter 2 are divided into two part.
The first part is the literature review. {\color{red}...} The other is the analysis of measurement data from Yang's team.
Most of those are the drain current of nanowire sweeping along the gate voltage (Id-Vg curves).
We present some of the raw data and the analysis results in this part.

Chapter 3 gives a brief description of nanowire structure.
It is then followed by some analysis of data from biology experiment and electrical measurement.
The biology experiment are done by Prof.Yang's team while electrical measurement are performed by ourselves.

Chapter 4 is an ``accessory''.
We construct an discrete circuit which was designed for ion-sensitive field-effect transistor (ISFET) \cite{SF1}.
The purpose of this process is to practice the constant current method.
The outcomes are deficient and it is its reference value which we spotlight.

{\color{red}Chapter 5}


\begin{figure}[!htbp]
    \centering
    {\fontfamily{pag}\selectfont\textbf{
        \def\svgwidth{5.0cm}
        \fontsize{6}{7}\selectfont
        \input {images/drawing-1.pdf_tex}
    }}
    \fontsize{6}{7}\selectfont
    \caption{Design Flow}
    \label{fig:designFlow}
\end{figure}


% This circuit has two operation mode.
% One is the ``Gate Voltage Tracing (GVT)" mode, and the other is the ``Current Variance Measure (CVM)" mode.



%
% {\color{red}A better measure method}
%
%  The concentration changing of the testing solution effect the intrinsic characteristics of nanowire.
% There are a variety of method for quantifying the affect.
% The most common ones are finding drain-source current and nanowire drain-to-source resistance.
% The method that Yang's team employed is to find several Id-Vg sweeping curves and see the rising/falling trend of them (Fig.~\ref{fig:IdVgandgbsId} (a)).
%  By the method of Yang's team, we had an intuition using transconductance instead of drain-to-source resistance may give a more {\color{red}usful} result.
% {\color{red}Because the }
%
%
% In the beginning, we adopted Yang's method and made some modification to it.
% By finding the derivative of Id with respect to Vg, we have the relation between Id and transconductance of nanowire (Fig.~\ref{fig:IdVgandgbsId} (b)).

% In our biosensor system, nanowire is treated as a MOSFET.
% Its gate bias under a specific voltage source
% And the bio-signal is viewed as small voltage signal input to the gate.
%
%
%  with its drain-source current ($I_{ds}$)  biased by a pmos current source.
% When a measurement event happens (such as a DNA concentration variation), the transconductance of nanowire changes and induces a current variance.
% This variance is converted into an amplified voltage signal.
% After the measurement, a feedback circuit pulls up/down the nanowire gate-source voltage ($V_{gs}$) to set $I_{ds}$ to the initail value.




%
% The different concentration of the measuring solution induce the change of nanowire transcconduction.
% By keeping bias voltage (drain, source, gates) same, the transconduction chage directly effect the drain-source current.
% The current variance signal is converted into voltage signal and is amplified.
% After each measurement, the drain-source current of nanowire returns to the constant value by a feedback circuit, which adjusts the gate-source voltage.
% the current variance induced by the difference of the solution concentration
% The other is the process variation. Since



