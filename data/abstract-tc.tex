\thispagestyle{empty}
\chapter*{\centerline{\fontsize{16pt}{\baselineskip}{中 \quad 文 \quad 摘 \quad 要}}}
\addcontentsline{toc}{chapter}{中文摘要}
\fontsize{14pt}{\baselineskip}

\quad 矽基體奈米線(以下簡稱奈米線)乃ㄧ有趣並經過充分研究的一維奈米結構物。
自從於 2001 年該元件被引入生物量測領域,奈米線就被賦予高度期望,能成為具有高靈敏度、即時性和不需生物標記等優勢的生物分子感測元件。
雖然如此,目前仍有有物性和化性上的因素,限制了奈米線的良率和實用性。
如今,許多研究採用了積體電路的技術,設計出在概念上和目的上不同的電路,來解決奈米線所遇到的問題和應付特定的需求。

我們使用由交大楊裕雄老師的奈米線元件和一部分量測資料,設計出一套元件訊號讀取電路。
本篇研究首先進行生物實驗和電性量測的數據分析,接著根據分析結果來訂出電路的規格和進行電路設計。

我們的電路可以進行直流掃描量測(DC-sweep)和暫態量測(Transient measurement)。
並且,藉由結合此二量測,我們提出一套元件量測方式。
這個量測方式被認為具有減輕因製程變異性而導致的元件差異問題(Device Variability)的能力。
目前,該量測方式多採用手動操作,我們希望日後能引入數位電路於系統性架構,使量測能夠自動化。


\clearpage % To prevent the page number missing