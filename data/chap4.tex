\chapter{Discrete Circuitry Design}
This chapter contains the discrete circuit which has been briefly reviewed in section \ref{section:SF}.
We build this circuit to practice the constant current method.


\section{Circuit Description}
The circuit is divided into two sections: the circuit body and the biasing current source ($I_b$).

The circuit body section is a source follower structure.
The input of the circuit is at the gate ($G$) of the core transistor, where the output is at its source ($S$).

The $I_D$ of the transistor is always equal to $I_b$ since there is no other current path.
And the $V_{DS}$ is always equal to the potential difference across the resistor ($R_b$).
Two OP is connected as unity gain buffer.
They connected serially with $R_b$ and cause the voltage at drain end $D$ follows the voltage at $S$.

As for the $I_b$, it is in fact a current scale down circuit.
By concerning the OP as ideal, the node $i_{out}$ is same with $i_0$.
And the output current is $N$-fold to the input current ($I_s$).

\section{Transforming the design from p-type measuring into n-type measuring}
Different from the ISFET in \cite{SF1}, our element is n-type.
We transform the circuit as in Fig.

\section{Discrete Element}
For the OP, we use tlc2264 made by Texas Insrument (TI).
This OP element can perform rail-to-rail 

\section{Measurement Results and Conclusion}

